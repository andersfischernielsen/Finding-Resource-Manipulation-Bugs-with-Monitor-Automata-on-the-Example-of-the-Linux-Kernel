\section{Future Work}

Implementing other bug checkers using monitor templates would allow EBA to detect these bug types. An implementation of a use-after-free bug checker could specifically prove very useful in detecting security risks in programs. This type of memory bug has been deemed as the cause for 70\% of the critical security vulnerabilities found in the popular Chromium browser, which is the base for Google's Chrome browser \cite{google-use-after-free}. This goes to show that this type of bug is prevalent in the wild and detecting such bugs in the Linux kernel would benefit the community surrounding the Linux kernel.

\newpar The addition of more effect types to the EBA framework would allow defining a a bigger range of monitor templates, in turn enabling the detection of more bug types. Such effects could show effects of multi-threaded code, allowing the definition of bug checkers for concurrency errors. 

\newpar The effect inference of the EBA framework could be improved upon in order to reduce the imprecision in the inferred effects. Currently, program points can have a wide range of effects after inference, leading to false positives. The inference of especially \textit{may} effects could possibly be improved in order to limit the exploration of program points which have none of the inferred possible effects when executed in practice. 

\newpar The GCC compiler supports compiler extensions and the Linux kernel project utilizes some of these, specifically \textit{Static Assertions}, added in C11 which have been implemented since since GCC 4.6\footnote{See \textit{Programming Languages - C} \cite{ISO:2011:IIIb}} and \textit{Assembler Instructions with C Expression Operands}, an extension available since GCC 3.1\footnote{See \textit{Assembler Instructions with C Expression Operands} \cite{GCC:3.1}}. The use of these produces compiler output which EBA does not support and the analysis will therefore fail. Support for the output of recent versions of GCC would fix this issue. 

\newpar Symbolic execution \cite{symbolic-execution} could be utilized in order to check the feasability of paths found in errors. Verifying that an execution path leading to a bug is actually reachable would increase the precision of bug checkers.  

\newpar The number of false positives reported by the implementation of a double-unlock monitor template in this thesis should be improved upon. Johnson, Song, Murphy-Hill and Bowdidge \cite{false-positives} have concluded in their research that false positives is a key factor in developers choosing to not use static analysis tools. They note that future static analysis should strive to improve upon this problem in order to see increased use. The number of false positives in loops for the implementation presented in this thesis could be implemented through extensions to the output of EBA to expose information on loops in the effect-CFG or by implementing sophisticated filtering of bug checker output.